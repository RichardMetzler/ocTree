\chapter{Concept}
\label{sec:concept}

 % SELECTION APPLICATION, QUADTREE, USERSTUDY, VERGLEICHS-MAß

The goal of this work was to describe and implement a framework that allows for meaningful statements about differences between user selected regions of importance and computer generated mesh saliency maps to be made. It also included such a comparison on a basic, conceptual level. Based on these abstract tasks given, the workload and how I approached it, can be described by the following milestone-like, high level requirements.

\begin{enumerate}
	\item implement a selection application for the V2C's projection installation
	\begin{enumerate}
		\item spatial indexing of 3D data
		\item selection process
	\end{enumerate}
	\item conduct a user study to acquire data for a comparison
	\item conceptualise a measure of differences between the data sets
\end{enumerate}

In this section, I will go through these requirements and describe in more detail what specific challenges they entailed and how I went about implementing solutions to them. I will also describe the underlaying concepts I chose to use and how I amended them to better fit this work where needed.

	\section {Implementing a selection application for the V2C's projection installation}
	\label{sec:implementing_selection_application_v2c}
	% LRT = leibinz supercomputing centre(?)
One of the main challenges of this work was developing a piece of C++ software that allows users in the five-sided projection installation of the Leibniz Supercomputing Centre to select vertices of 3D objects using the existing soft- and hardware components at hand there. Designing, implementing and adjusting this software to being executable as a multi-threaded client-server application in said projection installation, was a challenging and time-consuming aspect of this work. For the rest of this work, this piece of software will be refferend to as \textit{selection application}. For details on its implementation, see section \ref{sec:selection_application}.

		\subsection{Spatial indexing of 3D data}
		\label{sec:impl_spatial_indexing_3d}
3D objects and data in general are read as lists of coordinates by computers. Common 3D file formats such as .OBJ, .FBX and .STL contain the same data in similar structures, using multiple lists of different kinds of geometric information. While they these formats vary in the range of information they can hold, they all represent at least the following types of essential data. \textit{Vertices}, or a three tuple of float values describing x, y and zu coordinates and \textit{faces}, or basic triangles consisting of three vertices. Other optinal information that can be represented include vertex normals, texture coordinates and more complex features such as assigned materials, animations and armature objects. Figure \ref{fig:obj_content} shows an excerpt of a 3D file in .OBJ format, viewed in a simple text editor (gedit).

\begin{figure}[htb]
  \centering
  \includegraphics[width=.6\textwidth]{obj_content.png}\\ % PNG-File
  \caption{Notation in .OBJ format}\label{fig:obj_content}
\end{figure}

All these types of information share the following properties which were of relevance for this work. They are contiguous lists of lines, each line representing one instance of the data type indicated in the beginning of the line. Figure \ref{fig:obj_content} depicts an excerpt of an .OBJ file ocntaining vertices (lines starting with v), vertex texture coordinates (lines starting wit vt) and faces (lines starting with f).

These lists are not ordered. Depending on the modelling process, the vertices, faces and all the other attributes can be presented in an completely arbitrary order which hold very little information about the actual, geometric features of the object. The spatial position of any given vertex in relation to the entire object can not be retrieved from this type of notation. This created the demand for spatially indexing of 3D data in the scope this work.

I decided to implement the concept of Octrees \cite{Octree} because of its convenient characteristics as well as prior, personal working experience with \textit{quadtrees}. In an octree, the geometric size of the smallest possible leaf node can be determined in direct relation to the object to be indexed and, at any level, all leafs and nodes will be of the same size. This is highly useful in radius-based proximity-requests for multipl reasons as described in section \ref{sec:addVerticesToSelectionByCoordinates()}.

One of the most common queries that ar performed on 3D data are proximity queries. This means that vertices that are located within a given radius around an input query-coordinate are to be retrieved. The trivial and obviously highly inefficient solution to such a problem would be to iterate through the entire list of vertices and check for those who fulfill this query condition. This is where spatial indexing structures such ad octree come into play. The concept of octrees is highly recursive and can realize all the most common types of queries (including search queries) in logarithmic time.

An octree is a set of nodes that store references to one another. The highest-level node is called the root node, bottom level nodes are usually referred to as leaf nodes or leafs. Every node that is not a leaf node has eight children-nodes which, in a spatial sense, make up the entire space of their parent node. Such non-leaf nodes can also be described as roots of subtrees.
The most important parameter for an octree is the maximum allowed number of vertices per leaf node. Once the dimensions in x-,y - and z-direction of the 3D object to be indexed have been determined, the root node of the octree will store every vertex until said maximum allowed number of vertices is reached. Up until that point, the root node was still a leaf. It is now a root no longer and will create eight new nodes (its child nodes) and store references to them. This process is repeated for recursively for every new node until every leaf node holds less than the maximum allowed number of vertices per leaf node. Note that in many implementations of the octree concept, every leaf node needs to be at the same level. For this work, I decided to implement the non-balanced version of an octree where this is not the case.

		\subsection{Selection process}
		\label{sec:selection_process}
