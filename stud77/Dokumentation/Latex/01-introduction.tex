\chapter{Introduction}

Beginning in the 1950's, virtual reality technology [\cite{steuer1992defining}] has been continuously researched and improved and its professional relevance is becoming ever more present today. There is a plentitude of recent works showing that it bears great potential and positive possible contributions to architecture and construction [\cite{sampaio2014application}, \cite{le2015social}, \cite{stouffs2013happening}], healthcare and psychotherapy [\cite{baus2014moving}, \cite{merians2014rehabilitation}, \cite{de2014healthcare}], engineering and industrial design [\cite{marks2014towards}, \cite{wendrich2016hybrid}], gaming and home entertainment [\cite{valente2016live}, \cite{zyda2005visual}] and education [\cite{merchant2014effectiveness}, \cite{ott2015literature}]. One must also consider that this technology can help gaining new insights and open up new perspectives into greater, more abstract matters of social, environmental and economic manner [\cite{ovtcharova2015innovation}, \cite{nguyen2016applying}]. 

With resources such as memory and computing power becoming more and more available at ever-increasing rates, 3D objects and their mesh representations are constantly growing in complexity and size, in terms of shaders, texture maps as well as the sheer number of vertices. Still, many professional applications revolving around interaction with such models require means of displaying them in real-time without significant perceived loss of quality to ensure a smooth and fast workflow. This is where mesh simplification and segmentation plays an important role [\cite{wei2010feature}], [\cite{shaffer2001efficient}], [\cite{zhao2012saliency}] etc.

This issue becomes even more pressing in a professional, commercial context where access to state of the art, high-performance graphic processing units or render farms is not a given for everybody. With less computing power available, means of user-oriented, real-time rendering are of vital importance to a fast and unimpeded way of working on 3 dimensional assets.

Little has been done so far concerning research on \textit{mesh saliency} on a vertex level in a virtual reality environment. Whether saliency maps computed via known methods affect user behavior immersed in VR scenes at all, or to which extent has not been investigated in an object manipulation scenario. Furthermore, as of now, the effect on perceived visual quality of saliency-based object simplification as well as user behavior when given the opportunity to declare salient regions themselves is barely touched on at all.

The basic idea of the virtual reality (VR) CAVE [\cite{cruz1993surround}] is to create an interactive, immersive virtual reality environment for users by setting up multiple projectors and tracking sensors in a room with three to six solid walls that are also suitable be used as a projection surface. The users wear a pair of stereo shutter glasses which are synchronized with the projectors and thus can separate two images for the spectator - one for each eye. The glasses are equipped with some type of tracking system, for example an electromagnetic one, so that their exact position, orientation and tilt can be captured in real-time, allowing the computation of the users perspective in 3D scene at any time. Based on this perspective, the projectors use the walls as projection surfaces and throw live images resembling what the user would see if he/she were physically in the virtual scene onto the walls.

Depending on the number of projectors used, this usually invokes the issue of the user's shadow greatly impairing the immersion. In order to solve this problem, a CAVE must be carefully planned and set up, usually as an entirely separate room within a building. The projectors are then installed behind the walls of that room and their projection is cast onto them from the back.

Development of the CAVE model started in 1992 and has since been a continuous effort towards a professional VR solution that is able to eliminate most common problems that other common setups tended to cause back then - and still do - like the need for wearing an uncomfortable VR-headset attached to a cable, limiting the mobility of the user.
