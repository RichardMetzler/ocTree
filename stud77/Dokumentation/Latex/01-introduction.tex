\chapter{Introduction}
\label{sec:introduction}

Beginning in the 1950's, virtual reality technology \cite{steuer1992defining} has been continuously researched and improved and its professional relevance is becoming ever more present today. There is a plentitude of recent works showing that it bears great potential and positive possible contributions to architecture and construction \cite{sampaio2014application}, \cite{le2015social}, \cite{stouffs2013happening}, healthcare and psychotherapy \cite{baus2014moving}, \cite{merians2014rehabilitation}, \cite{de2014healthcare}, engineering and industrial design \cite{marks2014towards}, \cite{wendrich2016hybrid}, gaming and home entertainment \cite{valente2016live}, \cite{zyda2005visual} and education \cite{merchant2014effectiveness}, \cite{ott2015literature}. One must also consider that this technology can help gaining new insights and open up new perspectives into greater, more abstract matters of social, environmental and economic manner \cite{ovtcharova2015innovation}, \cite{nguyen2016applying}. 

With resources such as memory and computing power becoming more and more available at ever-increasing rates, 3D objects and their mesh representations are constantly growing in complexity and size, in terms of shaders, texture maps as well as the sheer number of vertices. Still, many professional applications revolving around interaction with such models require means of displaying them in real-time without significant perceived loss of quality to ensure a smooth and fast workflow. This is where mesh simplification and segmentation plays an important role \cite{wei2010feature}, \cite{shaffer2001efficient}, \cite{zhao2012saliency} etc.

This issue becomes even more pressing in a professional, commercial context where access to state of the art, high-performance graphic processing units or render farms is not a given for everybody. With less computing power available, means of user-oriented, real-time rendering are of vital importance to a fast and unimpeded way of working on 3D assets.

Little has been done so far concerning research on \textit{mesh saliency} on a vertex level in a virtual reality environment. Whether saliency maps computed via known methods affect user behavior immersed in VR scenes at all, or to which extent has not been investigated in a vertex selection scenario. Furthermore, as of now, the effect on perceived visual quality of saliency-based object simplification as well as user behavior when given the opportunity to declare salient regions themselves is barely touched on at all.

For the selection of parts of objects which seem important to beholders, the Virtual Reality and Visualisation Centres five-sided projection installation at the Leibniz Supercomputing Centre in Munich was available \ref{v2c}. This installation creates interactive, immersive virtual reality environments via multiple projectors and tracking sensors. Users only need to wear a lightweight pair of stereo shutter glasses that are synchronised with the projectors and thus can seperate two images for the spectator - one for each eye. The glasses are equipped with an electromagnetic tracking system so that their exact position, orientation and tilt can be captured in real-time, allowing the computation of the users perspective in 3D scene at any time. Based on this perspective, the projectors use the walls as projection surfaces and throw live images resembling what the user would see if he/she were physically in the virtual scene onto the walls. The projection installation grants a virtual reality experience which is enhanced by the fact that the user only needs to wear a pair of glasses instead of a fully sized headset. From a user perspective, another advantage of such a setup is the fact that there are no cables connected to the glasses which can evoke a feeling of inhibition or the constant worry of stumbling over and accidentally damaging them.

Finally, it is worth noting that the effect on perceived quality of objects in 3D scenes that can possibly achieved through saliency-based simplification is not limited to virtual reality applications. Long-term goals of efficiency and optimisation will continue to be accompanied by the need for semi-automatic complexity reduction of objects without a great loss of visual appeal, regardless of the type of media they will be presented on. So the best case outcome of this work is to find possible approaches to identifying segments of objects which are of high importance to the average beholder based on the immersive nature of the selection process of said segment.

This work is segmented into three major tasks. First, a piece of software that allows real-time interaction in an immersive virtual reality environment had to be implemented. This so-calld \textit{selection application} had to let users select and deselect vertices of 3D objects in an easy, fast way and provide clear visual feedback on what is currently selected and what is not. Then, a user study was conducted. 32 Users were invited to use the \textit{selection application} in the five-sided projection installation of the V2C \cite{v2c} and select parts of three 3D objects that they deemed \textit{important}. With this data, a comparison between automatically calculated \textit{importance} (\textit{mesh saliency}) maps and \textit{user saliency} maps could begin. The last step was the to start such an evaluation. A measure of difference was conceptualised for this work, trying to describe how much user selections differ from what parts of 3D objects are deemed \textit{interesting} by automatic computation. Based on this measure as well as structured observations of the resulting maps, this evaluation and discussion began.

The measure of difference showed that, for the three objects used during the user study, user selections differed from computed \textit{importance maps} by roughly 23\%, 27\% and 38\%. However, the expresiveness of these ratios can be doubted because the measure of difference, being a first quick suggestion, lacks expresiveness as discussed in section \ref{sec:measure_of_difference}. Furthermore, observations showed that, while there are clear similarities in what is deemed \textit{important} by computation and users, individual, highly object-dependant differences are to be expected.

\begin{figure}[htb]
  \centering
  \includegraphics[width=0.5\textwidth]{selection_cave.png}
  \caption{Schematic depiction of how the \textit{selection application} works from a user perspective}
  \label{fig:intro_pic}
\end{figure}

Figure \ref{fig:intro_pic} shows the application that was implemented for this work. Details on its implementation can be found in chapter \ref{sec:selection_application}, the general structure and conceptual part of this work is described in chapter \ref{sec:concept}. For more related work on the topic, see chapter \ref{sec:related_work}.

Details on the user study can be found in chapter \ref{sec:user_study_chapter}, its results are presented and discussed in chapter \ref{sec:results_and_discussion}. Lastly, for a quick summery of this work as well as an outlook on possible future work, see chater \ref{sec:conclusion_and_future_work}.

